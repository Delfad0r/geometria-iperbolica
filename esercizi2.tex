\section*{Esercizi del 28 marzo}
\subsection*{Introduzione teorica}
\begin{definition*}
Sia $\Gamma<\Isom(\HH^n)$ un sottogruppo. Fissiamo un punto $x\in\HH^n$. Definiamo $L(\Gamma)$ come l'insieme dei punti limite in $\del\HH^n$ dell'orbita $\Gamma\cdot x$. Poniamo inoltre $O(\Gamma)=\del\HH^n\setminus L(\Gamma)$.
\end{definition*}

Mostriamo che questa definizione è ben posta (ossia non dipende dalla scelta del punto $x$).
\begin{proposition*}
Siano $x,x'\in\HH^n$. Sia $y\in\del\HH^n$ un punto limite dell'orbita $\Gamma\cdot x$. Allora $y$ è anche un punto limite dell'orbita $\Gamma\cdot x'$.
\end{proposition*}
\begin{proof}
Per ipotesi esiste una successione di isometrie $\{g_i\}_{i\ge 0}\subs\Gamma$ tali che $g_i(x)\to y$. Essendo $g_i$ un'isometria di $\HH^n$, vale $\dist(g_i(x),g_i(x'))=\dist(x,x')$, pertanto anche $g_i(x')\to y$.
\end{proof}

È evidente che $L(\Gamma)$ è un chiuso $\Gamma$-invariante. Inoltre abbiamo la seguente.
\begin{proposition*}
Supponiamo che $\Gamma$ non sia elementare. Sia $S\subs\del\HH^n$ un chiuso non vuoto $\Gamma$-invariante. Allora $L(\Gamma)\subs S$.
\end{proposition*}
\begin{proof}
Sia $K\subs\overline{\HH^n}$ l'inviluppo convesso di $S$. Poiché $\Gamma$ non è elementare, $S$ contiene almeno due punti, dunque $K\cap\HH^n$ è non vuoto. Scegliamo un $x\in K\cap\HH^n$; poiché $S$ è chiuso e $\Gamma$-invariante, lo stesso vale per $K$. Ma allora l'insieme dei punti limite di $\Gamma\cdot x$ che giacciono in $\del\HH^n$ (ossia $L(\Gamma)$ è contenuto in $K\cap\del\HH^n=S$, da cui la tesi.
\end{proof}

\begin{proposition*}
Supponiamo che $\Gamma$ agisca su $\HH^n$ in modo libero e propriamente discontinuo. Allora $\Gamma$ agisce liberamente anche su $O(\Gamma)$.
\end{proposition*}
\begin{proof}
È sufficiente mostrare che tutti i punti fissi di elementi di $\Gamma$ giacciono in $L(\Gamma)$. Sia $g\in\Gamma$; distinguiamo due casi.
\begin{itemize}
\item Se $g$ è iperbolico, considerando il modello del semispazio $H^n$ si vede immediatamente che i due punti fissi di $g$ sono anche punti limite di $\langle g\rangle$.
\item Se $g$ è parabolico, consideriamo una qualunque orbita $\langle g\rangle \cdot x$. Poiché $\overline{\HH^n}$ è metrizzabile e compatto, necessariamente questa orbita ammette un punto limite, il quale risulta fissato da $g$; ma $g$ ha un unico punto fisso, che dunque è anche un punto limite.\qedhere
\end{itemize}
\end{proof}

\begin{definition*}
Sia $K\subs\overline{\HH^n}$ un convesso chiuso contenente almeno due punti (dunque non tutto contenuto in $\del\HH^n$). Definiamo l'applicazione $\map{\rho_K}{\overline{\HH^n}}{K}$ come segue:
\begin{itemize}
\item se $x\in\HH^n$, allora $\rho_K(x)$ è il punto di $K\cap\HH^n$ di minima distanza da $x$;
\item se $x\in\del\HH^n$, allora $\rho_K(x)$ è l'unico punto di $K$ giacente sulla minima orosfera centrata in $x$ che interseca $K$.
\end{itemize}
\end{definition*}
Osserviamo che questa definizione è ben posta poiché $K$ è convesso e chiuso. Inoltre si verifica facilmente che vale l'uguaglianza $\rho_{g(K)}\circ g=g\circ\rho_K$ per ogni isometria $g$ di $\HH^n$.

\begin{proposition*}
La restrizione
\[
\map{\rho_K}{\HH^n\cup(\del\HH^n\setminus K)}{\HH^n}
\]
è continua.
\end{proposition*}
\begin{proof}
Utilizziamo il modello del disco. Sia $x\in\HH^n\cup(\del\HH^n\setminus K)$. Poiché $x\not\in\del\HH^n\cap K$, sicuramente $\rho_K(x)\in\HH^n$. A meno di isometria, possiamo supporre che $\rho_K(x)=0$. Sia ora $y\in\HH^n\cup(\del\HH^n\setminus K)$. Mostreremo che $\norm{\rho_K(y)}_E\le\norm{y-x}_E$, dove $\norm{-}_E$ indica la norma euclidea sul disco: questo sarà sufficiente per concludere.

Se $\rho_K(y)=0$ la disuguaglianza è sicuramente verificata, dunque possiamo supporre $\rho_K(y)\neq 0$. Osserviamo che tutto il segmento euclideo (che è anche un segmento iperbolico) $[0,\rho_K(y)]$ è contenuto in $K$, essendo $K$ convesso. Poiché $0$ è il punto di $K$ più vicino a $x$, allora necessariamente l'angolo fra $[0,\rho_K(y)]$ e $[0,x]$ è ottuso, dunque $\scal{x}{\rho_K(y)}\le 0$ (questa disuguaglianza è vera anche se $x=0$, nel qual caso l'angolo citato non è ben definito). Allo stesso modo, spostando $\rho_K(y)$ in $0$ e ricordando che le isometrie sono (anti)conformi, anche l'angolo fra $[\rho_K(y),0]$ e il segmento iperbolico fra $\rho_K(y)$ e $y$ è ottuso, dunque a maggior ragione anche l'angolo fra $[\rho_K(y),0]$ e $[\rho_K(y),y]$ (segmento euclideo) lo è. Segue che $\scal{\rho_K(y)-y}{\rho_K(y)}\le 0$ (di nuovo, questa disuguaglianza è vera anche se $y=\rho_K(y)$). Combinando le due disuguaglianze trovate otteniamo che
\[
\norm{\rho_K(y)}^2_E\le\scal{\rho_K(y)}{y-x}\le\norm{\rho_K(y)}_E\cdot\norm{y-x}_E,
\]
dove abbiamo applicato la disuguaglianza di Cauchy-Schwarz. La tesi segue immediatamente.
\begin{figure}[h!]
\centering
\begin{tikzpicture}[scale=4]
\tkzDefPoint(0,0){o}
\tkzDefPointOnCircle[angle=120,center=o,radius=.8]\tkzGetPoint{x}
\tkzDefPointOnCircle[angle=30,center=o,radius=1]\tkzGetPoint{u_1}
\tkzDefPointOnCircle[angle=-50,center=o,radius=1]\tkzGetPoint{u_2}
\tkzDefPointWith[orthogonal](u_1,o)\tkzGetPoint{u_3}
\tkzDefPointWith[orthogonal](u_2,o)\tkzGetPoint{u_4}
\tkzInterLL(u_1,u_3)(u_2,u_4)\tkzGetPoint{u}
\tkzCalcLength[cm](u,u_1)
\tkzDefPointOnCircle[angle=130,center=u,radius=\tkzLengthResult cm]\tkzGetPoint{y}
\tkzDefPointOnCircle[angle=160,center=u,radius=\tkzLengthResult cm]\tkzGetPoint{py}
\tkzDefPointWith[orthogonal,K=.25](py,u)\tkzGetPoint{v}
\tkzFillAngles[fill=blue!25,size=.1](py,o,x v,py,o)
\tkzMarkAngle[size=.1](py,o,x)
\tkzMarkAngle[size=.1,mark=||](v,py,o)
\tkzDrawCircle[R,gray](o,1cm)
\tkzDrawArc[gray](u,u_1)(u_2)
\tkzDrawSegment[black,line width=.7pt](x,o)
\tkzDrawArc[black,line width=.7pt](u,y)(py)
\tkzDrawSegment[blue,line width=.7pt](o,py)
\tkzDrawSegment[blue!50,line width=.7pt, dashed](x,y)
\tkzDrawSegment[gray](py,v)
\tkzDrawPoints(o,x,y,py)
\tkzLabelPoint[below](o){$0$}
\tkzLabelPoint[above](x){$x$}
\tkzLabelPoint[above](y){$y$}
\tkzLabelPoint(py){$\rho_K(y)$}
\end{tikzpicture}
\end{figure}
\end{proof}

\begin{proposition*}
Supponiamo che $L(\Gamma)$ contenga almeno due punti e che $\Gamma$ agisca in modo propriamente discontinuo su $\HH^n$. Allora $\Gamma$ agisce allo stesso modo su $\HH^n\cup O(\Gamma)$.
\end{proposition*}
\begin{proof}
Sia $K\subs\overline{\HH^n}$ l'inviluppo convesso di $L(\Gamma)$; poiché $L(\Gamma)$ è un chiuso $\Gamma$-invariante, lo stesso vale per $K$. Consideriamo l'applicazione $\map{\rho}{\HH^n\cup O(\Gamma)}{\HH^n}$ definita come la restrizione a $\HH^n\cup O(\Gamma)$ di $\rho_K$. Per quanto abbiamo osservato, $\rho$ è continua e soddisfa $\rho\circ g=g\circ\rho$ per ogni $g\in\Gamma$.

Mostriamo che l'azione di $\Gamma$ su $\HH^n\cup O(\Gamma)$ è propriamente discontinua. Siano $y,y'\in\HH^n\cup O(\Gamma)$. Poiché l'azione di $\Gamma$ su $\HH^n$ è propriamente discontinua, esistono intorni $U$, $U'$ di $\rho(y)$, $\rho(y')$ rispettivamente tali che $U\cap g(U')\neq\emptyset$ solo per un numero finito di $g\in\Gamma$. Scegliamo $W=\rho^{-1}(U)$ e $W'=\rho^{-1}(U')$ come intorni, rispettivamente, di $y$ e $y'$. Si verifica facilmente che se $U\cap g(U')=\emptyset$ allora $W\cap g(W')=\emptyset$, da cui la tesi.
\end{proof}

\begin{theorem*}
Supponiamo che la varietà iperbolica completa $M=\HH^n/\Gamma$ abbia volume finito. Sia
\[
S=\bigcup_{g\in\Gamma\setminus\{\id\}}\Fix(g)\subs\del\HH^n.
\]
Allora $S$ è denso in $\del\HH^n$.
\end{theorem*}
\begin{proof}
Osserviamo innanzitutto che $\Gamma$ non è elementare, altrimenti $M$ avrebbe volume infinito. Notiamo poi che $\overline{S}\subs\del\HH^n$ è un chiuso non vuoto $\Gamma$-invariante, dunque contiene $L(\Gamma)$. Allo stesso tempo, poiché $\Gamma$ agisce su $O(\Gamma)$ senza punti fissi, necessariamente $S\subs L(\Gamma)$; in particolare, essendo $\Gamma$ non elementare, $L(\Gamma)$ contiene almeno due punti.

Supponiamo ora per assurdo che $S$ non sia denso in $\del\HH^n$. Poiché $L(\Gamma)\subs\overline{S}$, segue che $O(\Gamma)$ è non vuoto. Sia dunque $y\in O(\Gamma)$; poiché l'azione di $\Gamma$ su $\HH^n\cup O(\Gamma)$ è libera e propriamente discontinua, esiste un intorno $W\subs\HH^n\cup O(\Gamma)$ di $y$ tale che $W\cap g(W)=\emptyset$ per ogni $g\in\Gamma$ diverso dall'identità. Poiché $y$ ha un sistema fondamentale di intorni costituito da semispazi, possiamo supporre che $W$ sia un semispazio. Ma allora $\vol(M)\ge\vol(W)=\infty$, il che contraddice l'ipotesi.
\end{proof}

\subsection*{Esercizio 2.1}
Per assurdo, sia $\varphi\in\Isom(\HH^n)$ un'isometria che commuta con tutti gli elementi di $\Gamma$. Distinguiamo due casi.
\begin{itemize}
\item Se $\varphi$ è parabolica o iperbolica, per un Lemma visto a lezione segue che $\Fix(\varphi)=\Fix(g)$ per ogni $g\in\Gamma$ non banale, ossia tutti gli elementi di $\Gamma$ non banali hanno gli stessi punti fissi. Per un altro Lemma visto a lezione, ciò implica che $\Gamma$ è elementare, il che contraddice l'ipotesi di finitezza del volume di $M$.
\item Se $\varphi$ è ellittica, denotiamo con $S\subs\overline{\HH^n}$ il sottospazio dei punti fissi di $\varphi$ (si tratta di un sottospazio proprio e non vuoto). Sia $g\in\Gamma$ non banale; mostriamo che $\Fix(g)\subs S$.
\begin{itemize}
\item Se $g$ è parabolica, poiché $\varphi$ e $g$ commutano abbiamo che $\varphi(\Fix(g))=\Fix(g)$, ossia l'unico punto fisso di $g$ è fissato da $\varphi$, da cui $\Fix(g)\subs S$.
\item Se $g$ è iperbolica, poiché $\varphi$ e $g$ commutano abbiamo che $g(S)=S$. Consideriamo il modello del semispazio $H^n$, in cui i punti fissi di $g$ siano $0$ e $\infty$. Ricordando che $\varphi$ si scrive come $(x,t)\mapsto\lambda(Ax,t)$, si vede immediatamente che $S$ deve necessariamente essere un'iperpiano ortogonale a $\del H^n$, ossia $\infty\in S$. Scambiando $0$ e $\infty$ otteniamo che entrambi i punti fissi di $g$ appartengono a $S$.
\end{itemize}
Poiché $S$ è un sottospazio proprio, $S\cap\del\HH^n$ non può essere denso in $\del\HH^n$, il che contraddice il Teorema.
\end{itemize}

\subsection*{Esercizio 2.4}
La proiezione $\map{\phi}{\ZZ[i]}{\ZZ[i]/2\ZZ[i]}$ induce un omomorfismo di gruppi
\[
\map{\Phi}{\PSL(2,\ZZ[i])}{\PSL(2,\ZZ[i]/2\ZZ[i])}
\]
applicando $\phi$ a ogni entrata della matrice. Sia $\Gamma<\PSL(2,\ZZ[i])$ il nucleo di $\Phi$. Poiché $\PSL(2,\ZZ[i]/2\ZZ[i])$ è finito, $\Gamma$ ha indice finito in $\PSL(2,\ZZ[i])$. Osserviamo che
\[
\Gamma=\left\{\mattwo{1+2a}{2b}{2c}{1+2d}\in\PSL(2,\ZZ[i]):a,b,c,d,\in\ZZ[i]\right\}.
\]
(con lieve abuso di notazione, trattiamo gli elementi di $\PSL(2,\ZZ[i])$ come matrici invece che come classi di equivalenza).
\begin{itemize}
\item\textbf{$\Gamma$ non contiene elementi ellittici.} Supponiamo per assurdo che un elemento $\mattwo{1+2a}{2b}{2c}{1+2d}$ di $\Gamma$ sia ellittico, ossia abbia traccia reale minore di $2$ in modulo. Allora $2+2a+2d$ è un numero reale, intero, pari e minore di $2$ in modulo, dunque è necessariamente nullo. La condizione sul determinante diventa allora
\[
1=\det\mattwo{1+2a}{2b}{2c}{-1-2a}=-(1+4a+4a^2+4bc),
\]
ossia
\[
1+2a+2a^2+2bc=0,
\]
il che è assurdo (ad esempio guardando la parità della parte reale).
\item \textbf{La varietà iperbolica $\HH^3/\Gamma$ ha volume finito.} Dall'Esercizio 2.3 sappiamo che il gruppo $\PSL(2,\ZZ[i])$ ha un dominio fondamentale $D$ di volume finito. Poiché $\Gamma$ ha indice finito in $\PSL(2,\ZZ[i])$, allora esiste un dominio fondamentale (in senso lato) per $\Gamma$ di volume finito, che si ottiene prendendo l'unione dei domini $g\cdot D$ al variare di $g$ in un insieme di rappresentanti per $\PSL(2,\ZZ[i])/\Gamma$.
\item\textbf{La varietà iperbolica $\HH^3/\Gamma$ non è compatta.} Basta osservare che $\Gamma$ contiene l'elemento parabolico $\mattwo{3}{-2}{2}{-1}$.
\end{itemize}

\subsection*{Esercizio 2.5}
Poiché $M$ è compatta, tutti gli elementi di $\Gamma$ sono iperbolici. Dal Teorema segue immediatamente che $S$ è denso in $\del\HH^n$.