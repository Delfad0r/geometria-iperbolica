\tikzset{surface color/.style={fill={rgb:yellow,2.5;red,.5;white,4;gray,.5}}}
Osserviamo innanzitutto che se $i(a,b)\neq 0$ allora è sufficiente scegliere $c=a$, giacché $i(a,a)=0$. Supponiamo dunque che $i(a,b)=0$, e siano $\alpha$, $\beta$ curve semplici chiuse in posizione minimale che rappresentano rispettivamente $a$ e $b$; in particolare, $\alpha$ e $\beta$ hanno supporti disgiunti. Distinguiamo alcuni casi.
\begin{itemize}
\item \textbf{Supponiamo che $\alpha$ non sia separante.} Allora $S\cut \alpha$ è una superficie compatta connessa con due componenti di bordo.
\begin{itemize}
\item \textbf{Supponiamo che le due componenti di bordo appartengano alla stessa componente connessa di $S\cut\alpha\cut\beta$.} Allora esiste una curva $\gamma$ che interseca $\alpha$ esattamente una volta e non interseca $\beta$.
\end{itemize}
\end{itemize}