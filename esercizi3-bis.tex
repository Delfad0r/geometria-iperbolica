\tikzset{surface color/.style={fill={rgb:yellow,2.5;red,.5;white,4;gray,.5}}}
Osserviamo innanzitutto che se $i(a,b)\neq 0$ allora è sufficiente scegliere $c=a$, giacché $i(a,a)=0$. Supponiamo dunque che $i(a,b)=0$, e siano $\alpha$, $\beta$ curve semplici chiuse in posizione minimale che rappresentano rispettivamente $a$ e $b$; in particolare, $\alpha$ e $\beta$ hanno supporti disgiunti. Distinguiamo alcuni casi.
\begin{itemize}
\item \textbf{Supponiamo che $\alpha$ non sia separante.} Allora $S\cut \alpha$ è una superficie compatta connessa con due componenti di bordo.

\begin{center}
\begin{tikzpicture}
\pic[surface color,thick] {closed surface={radius=1.2cm,name=S,g=2,left=2}};
\begin{scope}[red,very thick]
\foreach \w in {S-1-t,S-1-b} {
    \draw[dashed,opacity=.4] (\w-1) to[right vertical torus arc] (\w-2);
    \draw (\w-1) to[left vertical torus arc] (\w-2);
}
\end{scope}
\node[red] at (-.4,0) {$\alpha$};
\node[anchor=west] at ($(S-r)+(.2,0)$) {$S\cut\alpha$};
\end{tikzpicture}
\end{center}

\begin{itemize}
\item \textbf{Supponiamo che le due componenti di bordo appartengano alla stessa componente connessa di $S\cut\alpha\cut\beta$.} Allora esiste una curva $\gamma$ che interseca $\alpha$ esattamente una volta e non interseca $\beta$.

\begin{center}
\begin{tikzpicture}
\pic[surface color,thick] {closed surface={radius=1.2cm,name=S,g=2,left=2}};
\path (S-1-t-1) to[left vertical torus arc] coordinate[pos=.3] (c1) (S-1-t-2);
\path (S-1-b-1) to[left vertical torus arc] coordinate[pos=.3] (c2) (S-1-b-2);
\begin{scope}[red,very thick]
\foreach \w in {S-1-t,S-1-b} {
    \draw[dashed,opacity=.4] (\w-1) to[right vertical torus arc] (\w-2);
    \draw (\w-1) to[left vertical torus arc] (\w-2);
}
\end{scope}
\begin{scope}[blue,very thick]
    \draw[dashed,opacity=.4] (S-2-b-1) to[right vertical torus arc] (S-2-b-2);
    \draw (S-2-b-1) to[left vertical torus arc] (S-2-b-2);
\end{scope}
\draw[very thick,green!80!blue] (c1) to[arc with diameter={90:-90:{1/cos(70)} and 1}] (c2);
\node[red] at (-.4,0) {$\alpha$};
\node[blue] at ($(S-2-b-1)!.5!(S-2-b-2)+(-.4,0)$) {$\beta$};
\node[green!80!blue] at ($(S-1-i-1)+(.2,-.2)$) {$\gamma$};

\pic[surface color,thick] at (7.5,0) {closed surface={radius=1.2cm,name=S2,g=2}};
\begin{scope}[red,very thick]
    \draw[dashed,opacity=.4] (S2-1-l) to[arc with diameter={10:-190:1 and .8}] (S2-l);
    \draw (S2-1-l) to[arc with diameter={10:170:1 and .8}] (S2-l);
\end{scope}
\begin{scope}[blue,very thick]
    \draw[dashed,opacity=.4] (S2-2-b-1) to[right vertical torus arc] (S2-2-b-2);
    \draw (S2-2-b-1) to[left vertical torus arc] (S2-2-b-2);
\end{scope}
\begin{scope}[green!80!blue,very thick]
\draw ($(S2-1-i-1)!.3!(S2-1-i-2)$) to [horizontal torus arc={0:180}] ($(S2-0-i-1)!.3!(S2-0-i-2)$);
\draw ($(S2-1-i-1)!.3!(S2-1-i-2)$) to [horizontal torus arc={0:-180}] ($(S2-0-i-1)!.3!(S2-0-i-2)$);
\end{scope}
\node[red] at ($(S2-l)+(-.2,.3)$) {$\alpha$};
\node[blue] at ($(S2-2-b-1)!.5!(S2-2-b-2)+(-.4,0)$) {$\beta$};
\node[green!80!blue] at ($(S2-1-i-1)+(.2,-.2)$) {$\gamma$};

\draw[thick,-implies,double equal sign distance,shorten <=.5cm,shorten >=.5cm] (S-r) -- (S2-l);
\end{tikzpicture}
\end{center}

Prendendo come $c$ la classe di isotopia di $\gamma$, otteniamo che
\[
i(a,c)=1\neq0=i(b,c).
\]
\item\textbf{Supponiamo che le due componenti di bordo appartengano a componenti connesse diverse di $S\cut\alpha\cut\beta$.} In questo caso $S\cut\alpha\cut\beta$ è unione disgiunta di due superfici $S_1$ e $S_2$, ciascuna con due componenti di bordo, una lungo $\alpha$ e una lungo $\beta$.

\begin{center}
\begin{tikzpicture}
\pic[surface color,thick] {closed surface={radius=1.2cm,name=S1,g=2,right=2}};
\pic[surface color,thick] at (3,0) {closed surface={radius=1.2cm,name=S2,g=2,left=2}};
\foreach \w/\col in {t/red,b/blue} {
    \draw[very thick,\col] (S1-2-\w-1) to[left vertical torus arc] (S1-2-\w-2) (S1-2-\w-1) to[right vertical torus arc] (S1-2-\w-2);
    \draw[very thick,\col,dashed,opacity=.4] (S2-1-\w-1) to[right vertical torus arc] (S2-1-\w-2);
    \draw[very thick,\col] (S2-1-\w-1) to[left vertical torus arc] (S2-1-\w-2);
}
\node at ($(S1-l)+(-.5,0)$) {$S_1$};
\node at ($(S2-r)+(.5,0)$) {$S_2$};
\node[red] at ($(S1-2-t-1)!.5!(S2-1-t-1)+(0,.2)$) {$\alpha$};
\node[blue] at ($(S1-2-b-2)!.5!(S2-1-b-2)+(0,-.2)$) {$\beta$};
\end{tikzpicture}
\end{center}

Osserviamo che $S_1$ e $S_2$ hanno genere almeno $1$, altrimenti $\alpha$ e $\beta$ coborderebbero un anello e sarebbero dunque isotope. È allora facile costruire una curva $\gamma$ la cui classe di isotopia $c$ soddisfa la tesi.
\begin{center}
\begin{tikzpicture}
\pic[surface color,thick] {closed surface={radius=1.2cm,name=S,g=3}};
\foreach \w/\col in {t/red,b/blue} {
    \draw[very thick,\col,dashed,opacity=.4] (S-2-\w-1) to[right vertical torus arc] (S-2-\w-2);
    \draw[very thick,\col] (S-2-\w-1) to[left vertical torus arc] (S-2-\w-2);
}
\draw[green!80!blue,very thick] ($(S-2-t-1)!.2!(S-2-t-2)$) to[out=0,in=180] ($(S-2-i-1)!.1!(S-2-i-2)$) to[out=0,in=180] ($(S-3-t-1)!.3!(S-3-t-2)$) to[arc with diameter={90:-90:{1/cos(70)} and 1}] ($(S-3-b-1)!.3!(S-3-b-2)$) to[out=180,in=-50] ($(S-2-i-1)!.35!(S-2-i-2)$) to[out=130,in=0,looseness=.4] ($(S-2-t-1)!.5!(S-2-t-2)$) to[out=180,in=50,looseness=.4] ($(S-1-i-1)!.35!(S-1-i-2)$) to[out=-130,in=0] ($(S-1-b-1)!.3!(S-1-b-2)$) to [arc with diameter={270:90:{1/cos(70)} and 1}] ($(S-1-t-1)!.3!(S-1-t-2)$) to [out=0,in=180] ($(S-1-i-1)!.1!(S-1-i-2)$) to[out=0,in=180] ($(S-2-t-1)!.2!(S-2-t-2)$);
\node[red] at ($(S-2-t-1)+(0,.3)$) {$\alpha$};
\node[blue] at ($(S-2-b-2)+(0,-.4)$) {$\beta$};
\node[green!80!blue] at ($(S-1-i-1)!.8!(S-1-i-2)$) {$\gamma$};
\end{tikzpicture}
\end{center}
Grazie al criterio del bigono, si verifica che $\alpha$ e $\gamma$ sono in posizione minimale, da cui
\[
i(a,c)=2\neq0=i(b,c).
\]
\end{itemize}
\end{itemize}