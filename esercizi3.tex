\section*{Esercizi dell'11 aprile}

\subsection*{Dehn twist}
Per fissare la notazione, ricordiamo come si definiscono i Dehn twist. Siano $S$ una superficie chiusa orientabile, $\map{\alpha}{S^1}{S}$ una curva semplice chiusa non banale. Sia $A=S^1\times[-1,1]$ con l'orientazione indotta da quelle standard di $S^1$ e $[-1,1]$. Fissiamo un intorno regolare di $\alpha$, ossia un embedding $\map{\psi}{A}{S}$ tale che $\psi(x,0)=\alpha(x)$ per ogni $x\in S^1$; scegliamolo in modo che $\psi$ preservi l'orientazione. Fissiamo infine una funzione $\map{f}{[-1,1]}{[0,2\pi]}$ tale che
$f(t)=0$ per $t\le-\frac{1}{2}$ e $f(t)=2\pi$ per $t\ge\frac{1}{2}$, e definiamo
\Map{\theta}{A}{A}{(e^{ix},t)}{(e^{i(x+f(t))},t).}
Possiamo ora definire il Dehn twist intorno a $\alpha$ come il diffeomorfismo $\map{T_\alpha}{S}{S}$ tale che
\[
T_\alpha(p)=\begin{cases}
p&p\not\in\psi(A)\\
(\psi\circ\theta\circ\psi^{-1})(p)&p\in\psi(A).
\end{cases}
\]
Come visto a lezione, la classe di isotopia di $T_\alpha$ non dipende dalla scelta dell'intorno regolare $\psi$ né della funzione $f$. Mostreremo inoltre nell'Esercizio 3.2 che la classe di isotopia di $T_\alpha$ non cambia invertendo l'orientazione di $\alpha$ o sostituendo $\alpha$ con una curva a lei isotopa: è dunque ben definito l'elemento $T_a\in\MCG(S)$ per $a\in\mathcal{S}$.

\newpage
\subsection*{Esercizio 3.1}
\tikzset{every plot/.style={smooth}}
\tikzset{bigon/.style={orange!60!yellow,opacity=.5}}
\tikzset{surface color/.style={fill={rgb:yellow,2.5;red,.5;white,4;gray,.5}}}
\newcommand{\definef}{
\tikzmath{
function f(\x) {
    if \x < -1 then {
        return 2 / (1 + exp(5));
    } else {
        if \x > 1 then {
            return 2 / (1 + exp(-5));
        } else {
            return 2 / (1 + exp(-\x * 5));
        };
    };
};
}
}
\newcommand{\exerciseIntersectLines}[3]{
\fill[white,name intersections={of={#1} and {#2},total=\t}] {\ifnum \t>0 \foreach \k in {1,...,\t} {(intersection-\k) circle (#3 pt) node[black,cross out,draw,scale=.5,very thick] {}}\fi};
}
\begin{tikzfadingfrompicture}[name=ex 3-1 fading 5]
\begin{scope}[scale=5]
\shade[left color=transparent!100,right color=transparent!0] (-1.2,-1) rectangle (-1,1);
\shade[left color=transparent!0,right color=transparent!100] (1,-1) rectangle (1.2,1);
\fill[transparent!0] (-1,-1) rectangle (1,1);
\end{scope}
\end{tikzfadingfrompicture}
\begin{lemma}\thlabel{intersection-number-dehn-twist}
Siano $a$, $b$ classi di isotopia di curve, con $b$ non banale, e sia $k\ge 0$. Allora
\[
i(a,T_b^k(a))=k\cdot i(a,b)^2.
\]
\end{lemma}
\begin{proof}
Siano $\alpha$, $\beta$ rappresentanti di $a$, $b$ in posizione minimale. Se $\alpha$ e $\beta$ non si intersecano la tesi è ovvia, dunque supponiamo che si intersechino almeno in un punto. Scegliamo un intorno regolare $U$ di $\beta$ abbastanza stretto da intersecare $\alpha$ in $i(a,b)$ archi disgiunti, ciascuno dei quali interseca $\beta$ esattamente una volta. Definiamo un rappresentante $\gamma$ della classe $T_b^k(a)$ come segue: consideriamo una curva $\alpha'$ parallela a $\alpha$, ottenuta traslando $\alpha$ lungo un suo intorno regolare, e poniamo $\gamma=T_\beta^k(\alpha')$. Possiamo orientare $\alpha$ e $\gamma$ in modo che siano coorientate e che entrando in $U$ si allontanino.


\begin{center}
\begin{tikzpicture}[scale=5]
\usetikzlibrary{shapes.misc}
\definef{}
\newcommand{\drawlines}{
\draw[black] (0,-1) -- (0,1);
\foreach \h/\dir/\del [count=\i] in {-.5/0/1,.1/180/-1,.6/0/1} {
    \draw[name path global=alpha\i,blue,thick,postaction={decorate},decoration={markings,mark=between positions 0 and 1 step .5cm with {\arrow[rotate=\dir]{stealth}}}] (-2,\h) -- (2,\h);
    \foreach \de [count=\j] in {0,-2} {
        \draw[name path global=gamma\i\j,red,thick,domain=-2:2
        ,postaction={decorate},decoration={markings,mark=between positions .1 and .9 step .5cm with {\arrow[rotate=\dir]{stealth}}}
        ] plot({\x},{.1*\del+\de+\h+f(\x)});
    }
}
};
%\begin{scope}
%\clip (-1.2,-1) rectangle (-1,1);
%\path[scope fading=west] (-1.2,-1) rectangle (-1,1);
%\drawlines
%\end{scope}
%\begin{scope}
%\clip (1,-1) rectangle (1.2,1);
%\path[scope fading=east] (1,-1) rectangle (1.2,1);
%\drawlines
%\end{scope}
\fill[gray!10] (-1,-1) rectangle (1,1);
\begin{scope}
%\clip (-1,-1) rectangle (1,1);
\path[scope fading=ex 3-1 fading 5,fit fading=false,use as bounding box] (-1.2,-1.1) rectangle (1.2,1.1);
\drawlines
\foreach \i in {1,2,3} \foreach \ii in {1,2,3} \foreach \j in {1,2} {
    \exerciseIntersectLines{alpha\i}{gamma\ii\j}{.6}s
}
\end{scope}
\draw[gray] (-1,-1) rectangle (1,1);
\node[red,above] at (-.8,-.33) {$\gamma$};
\node[blue,below] at (-.8,-.55) {$\alpha$};
\node[black,right] at (0,.3) {$\beta$};
\node[black!70] at (.8,.9) {$U$};
\end{tikzpicture}
\end{center}

È immediato verificare che $\alpha$ e $\gamma$ si intersecano esattamente in $k\cdot i(a,b)^2$ punti. È dunque sufficiente mostrare che $\alpha$ e $\gamma$ sono in posizione minimale, ossia che non formano bigoni. Se $i(a,b)=1$ allora $\alpha $ e $\gamma$ non possono formare bigoni (si intersecano sempre con la stessa orientazione), quindi supponiamo $i(a,b)\ge 2$. Supponiamo per assurdo che esista un bigono $D$, e siano $\hat\alpha$, $\hat\gamma$ i lati di $D$ che giacciono rispettivamente su $\alpha$ e $\gamma$. Distinguiamo alcuni casi.
\newpage
\begin{itemize}
\item Se $\hat\gamma$ è tutto contenuto in $U$, allora $\alpha$ e $\beta$ formano un bigono, ma ciò è impossibile, dato che sono in posizione minimale.
\begin{center}
\begin{tikzpicture}[scale=2.5]
\definef{}
\tikzmath{
\h1 = -.4;
\h2 = .3;
\r = (\h2 - \h1) / 2;
}

\newcommand{\setupfigure}{
\path[name path global=alpha] (1,\h2)-- (-1,\h2) arc (90:270:\r) -- (1,\h1);
\pgfinterruptboundingbox
\path[save path=\gammapath,domain=-1:1,name path global=gamma] plot({\x},{-1.0+f(\x)});
\endpgfinterruptboundingbox

\clip (-1.6,-.8) rectangle (.5,.7);
\fill[gray!10] (-1,-1) rectangle (1,1);
\draw[gray] (-1,-1) rectangle(1,1);
\begin{scope}[opacity=.5]
\clip (-1,-1) rectangle (1,1);
\draw[black,name path global=beta] (0,-1) -- (0,1);
\draw[blue] (-1,\h1) -- (1,\h1);
\draw[blue] (-1,\h2) -- (1,\h2);
\draw[red,use path=\gammapath];
\end{scope}
}


\begin{scope}[xshift=-1.3cm]
\setupfigure{}
\fill[bigon,intersection segments={of=alpha and gamma,sequence={L2--R2}}];
\begin{scope}
\clip (-1,-1) rectangle (1,1);
\draw[red,very thick,intersection segments={of=alpha and gamma,sequence={R2}}];
\draw[blue,very thick,intersection segments={of=alpha and gamma,sequence={L2}}];
\end{scope}
\draw[blue,very thick,dashed] (-1,\h2) arc (90:270:\r);
\exerciseIntersectLines{alpha}{gamma}{1.2};
\node[above,blue] at (-.4,.3) {$\hat\alpha$};
\node[right,red] at (.05,.0) {$\hat\gamma$};
\end{scope}

\begin{scope}[xshift=2.1cm]
\setupfigure{}
\fill[bigon,intersection segments={of=alpha and beta,sequence={L2--R2}}];
\begin{scope}
\clip (-1,-1) rectangle (1,1);
\draw[blue,very thick,intersection segments={of=alpha and beta,sequence={L2}}];
\draw[black,very thick,intersection segments={of=alpha and beta,sequence={R2}}];
\end{scope}
\draw[blue,very thick,dashed] (-1,\h2) arc (90:270:\r);
\node[above,blue] at (-.4,.3) {$\hat\alpha_1$};
\node[right,black] at (.0,.0) {$\hat\beta$};
\end{scope}

\draw[thick,-implies,double equal sign distance] (-.3,0) -- (.3,0);
\end{tikzpicture}
\end{center}
\item Se $\hat\alpha$ è tutto contenuto in $U$, allora di nuovo $\alpha$ e $\beta$ formano un bigono (ricordiamo che $\alpha$ e $\gamma$ sono parallele fuori da $U$).
\begin{center}
\begin{tikzpicture}[scale=2.5]
\definef{}
\tikzmath{
\h1 = -.5;
\h2 = .4;
\r = (\h2 - \h1) / 2;
}

\newcommand{\setupfigure}{
\path[save path=\alphapath,name path global=alpha] (1,\h2) -- (-1,\h2) arc (90:270:\r) -- (1,\h1);
\pgfinterruptboundingbox
\path[save path=\gammapath,domain=-1:1,name path global=gamma] plot({-\x},{\h2-.1+f(-\x)}) arc (90:270:{\r-.1}) -- plot({\x},{\h1+.1+f(\x)});
\endpgfinterruptboundingbox

\clip (-1.6,-.8) rectangle (.5,.7);
\fill[gray!5] (-1,-1) rectangle (1,1);
\draw[gray] (-1,-1) rectangle(1,1);
\begin{scope}[opacity=.5]
\clip (-2,-1) rectangle (1,1);
\draw[black,name path global=beta] (0,-1) -- (0,1);
\begin{scope}
\clip (-1,-1) rectangle (1,1);
\draw[blue,use path=\alphapath];
\draw[red,use path=\gammapath];
\end{scope}
\end{scope}
}


\begin{scope}[xshift=-1.3cm]
\setupfigure{}
\fill[bigon,intersection segments={of=alpha and gamma,sequence={L2--R2}}];
\begin{scope}
\clip (-1,-1) rectangle (1,1);
\draw[red,very thick,intersection segments={of=alpha and gamma,sequence={R2}}];
\draw[blue,very thick,intersection segments={of=alpha and gamma,sequence={L2}}];
\end{scope}
\begin{scope}
\clip (-2,-1) rectangle (-1,1);
\draw[blue,dashed,opacity=.5,use path=\alphapath];
\draw[red,dashed,very thick,intersection segments={of=alpha and gamma,sequence={R2}}];
\end{scope}
\exerciseIntersectLines{alpha}{gamma}{1.2};
\node[above,blue] at (-.3,.4) {$\hat\alpha$};
\node[right,red] at (-.2,.0) {$\hat\gamma$};
\end{scope}

\begin{scope}[xshift=2.1cm]
\setupfigure{}
\begin{scope}[opacity=.5]
\clip (-2,-1) rectangle(-1,1);
\draw[red,dashed,use path=\gammapath];
\end{scope}
\fill[bigon,intersection segments={of=alpha and beta,sequence={L2--R2}}];
\begin{scope}
\clip (-1,-1) rectangle (1,1);
\draw[blue,very thick,intersection segments={of=alpha and beta,sequence={L2}}];
\draw[black,very thick,intersection segments={of=alpha and beta,sequence={R2}}];
\end{scope}
\begin{scope}
\clip (-2,-1) rectangle (-1,1);
\draw[blue,dashed,very thick,intersection segments={of=alpha and beta,sequence={L2}}];
\end{scope}
\node[above,blue] at (-.3,.4) {$\hat\alpha_1$};
\node[right,black] at (.0,.0) {$\hat\beta$};
\end{scope}

\draw[thick,-implies,double equal sign distance] (-.3,0) -- (.3,0);
\end{tikzpicture}
\end{center}
\item Dunque $\hat\alpha$ esce da $U$ per poi rientrarvi. Analizziamo cosa succede vicino vicino al punto in cui $\hat\alpha$ esce da $U$.
\begin{center}
\begin{tikzpicture}[scale=5]
\definef{}
\tikzmath{
\h1 = .1;
\h2 = .3;
}
\begin{scope}
\clip (.2,-.1) rectangle (1.2,.5);

\path[name path=alpha,save path=\alphapath] (-1,\h1) -- (2,\h1) (-1,\h2) -- (2,\h2);
\path[name path=gamma,save path=\gammapath,domain=-1:2] plot({\x},{-2+\h2+.1+f(\x)});

\fill[gray!10] (-1,-1) rectangle (1,1);
\draw[gray] (-1,-1) rectangle (1,1);
%\foreach \f in {1,0} {
\begin{scope}
%\ifnum\f=1
%    \clip (1,-1) rectangle (2,1);
%    \path[scope fading=east] (1,-1) rectangle (1.2,1);
%\else
%    \clip (-1,-1) rectangle (1,1);
%\fi
\path[scope fading=ex 3-1 fading 5,fit fading=false] (0,0);
\fill[bigon,intersection segments={of=alpha and gamma,sequence={R3--L3[reverse]}}];
\begin{scope}
\clip[use path=\gammapath -- cycle];
\fill[bigon,purple!50!white] (-1,\h1) rectangle (2,\h2);
\end{scope}
\draw[use path=\alphapath,blue];
\draw[use path=\gammapath,red];
\draw[red,thick,intersection segments={of=alpha and gamma,sequence={R3}},postaction={decorate},decoration={markings,mark=between positions 0.1 and 1 step .4cm with {\arrow[scale=.8]{stealth}}}];
\draw[blue,thick,intersection segments={of=alpha and gamma,sequence={L3}},postaction={decorate},decoration={markings,mark=between positions 0.1 and 1 step .4cm with {\arrow[scale=.8]{stealth}}}];
\end{scope}
%}
\path[name path=alpha2] (-1,\h2) -- (1,\h2);
\exerciseIntersectLines{alpha}{gamma}{.5};
\end{scope}
\node[red,above] at (1.2,\h2+.1) {$\hat\gamma$};
\node[blue,below] at (1.2,\h2) {$\hat\alpha$};
\end{tikzpicture}
\end{center}
Osserviamo che la regione in rosa non può essere un bigono, in quanto il suo lato giacente su $\gamma$ è tutto contenuto in $U$, e abbiamo già escluso questa possibilità. Dunque il bigono è necessariamente la regione arancione, e $\hat\gamma$ è parallelo a $\hat\alpha$ fuori da $U$. Analizziamo ora cosa succede vicino al punto in cui $\hat\alpha$ e $\hat\gamma$ rientrano in $U$.
\begin{center}
\begin{tikzpicture}[scale=5]
\definef{}
\tikzmath{
\h1 = .0;
\h2 = .3;
}

\begin{scope}
\clip (-1.2,-.15) rectangle (-.2,.45);
\path[name path=alpha,save path=\alphapath] (-2,\h1) -- (1,\h1) (-2,\h2) -- (1,\h2);
\path[name path=gamma,save path=\gammapath,domain=-2:1] plot({\x},{\h1+.1+f(\x)});

\fill[gray!10] (-1,-1) rectangle (1,1);
\draw[gray] (-1,-1) rectangle (1,1);

%\foreach \f in {1,0} {
\begin{scope}
%\ifnum\f=1
%    \clip (-2,-1) rectangle (-1,1);
%    \path[scope fading=west] (-1.2,-1) rectangle (-1,1);
%\else
%    \clip (-1,-1) rectangle (1,1);
%\fi
\path[scope fading=ex 3-1 fading 5,fit fading=false] (0,0);
\begin{scope}
\path[name path global=bigonside,save path=\bigonside] (-2,\h1) rectangle (1,\h2);
\clip[intersection segments={of=bigonside and gamma,sequence={L1--R2--L3}}];
\fill[bigon,use path=\bigonside];
\end{scope}
\draw[use path=\alphapath,blue];
\draw[use path=\gammapath,red];
\draw[red,thick,postaction={decorate},decoration={markings,mark=between positions .2 and 1. step .4cm with {\arrow[scale=.8]{stealth}}},intersection segments={of=bigonside and gamma,sequence={R2}}];
\draw[blue,thick,postaction={decorate},decoration={markings,mark=between positions 0.1 and 1 step .4cm with {\arrow[scale=.8]{stealth}}}] (-2,\h1) -- (1,\h1);
\end{scope}
%}
\exerciseIntersectLines{alpha}{gamma}{.5};
\end{scope}
\node[red,above] at (-1.2,\h1+.1) {$\hat\gamma$};
\node[blue,below] at (-1.2,\h1) {$\hat\alpha$};
\end{tikzpicture}
\end{center}
Dopo essere entrati in $U$, $\hat\alpha$ e $\hat\gamma$ si allontanano, dunque non è possibile che la prima intersezione di $\hat\gamma$ con $\alpha$ sia il secondo estremo di $\hat\alpha$; è pertanto impossibile che $\hat\alpha$ e $\hat\gamma$ siano lati di un bigono.\qedhere
\end{itemize}
\end{proof}
Sia $b\in\mathcal{S}$ una classe di isotopia non banale, e $\map{\beta}{S^1}{S}$ una curva che la rappresenta. Mostriamo che esiste $a\in\mathcal{S}$ tale che $i(a,b)\neq 0$.
\begin{itemize}
\item \textbf{Supponiamo che $\beta$ sia separante.} In questo caso $S\cut\beta$ è unione disgiunta di due superfici $S_1$ e $S_2$, ciascuna con una componente di bordo. Naturalmente $S_1$ e $S_2$ hanno genere almeno $1$, altrimenti $\beta$ sarebbe omotopicamente banale.

\begin{center}
\begin{tikzpicture}
\pic[surface color,thick] {closed surface={radius=1.2cm,name=S1,g=1,right=1}};
\pic[surface color,thick] at (3,0) {closed surface={radius=1.2cm,name=S2,g=2,left=1}};
\begin{scope}[red,very thick]
\draw (S1-1-i-1) to[left vertical torus arc] (S1-1-i-2) (S1-1-i-1) to[right vertical torus arc] (S1-1-i-2);
\draw (S2-0-i-1) to[left vertical torus arc] (S2-0-i-2);
\draw[dashed,opacity=.4] (S2-0-i-1) to[right vertical torus arc] (S2-0-i-2);
\end{scope}
\node[red] at ($(S1-1-i-2)!.5!(S2-0-i-2)+(0,-.2)$) {$\beta$};
\node at ($(S1-l)+(-.5,0)$) {$S_1$};
\node at ($(S2-r)+(.5,0)$) {$S_2$};
\end{tikzpicture}
\end{center}

Allora possiamo prendere come $a$ la classe di isotopia della curva $\alpha$ ottenuta come in figura.

\begin{center}
\begin{tikzpicture}
\pic[surface color,thick] {closed surface={radius=1.2cm,name=S,g=3}};
\begin{scope}[red,very thick]
\draw (S-1-i-1) to[left vertical torus arc] (S-1-i-2);
\draw[dashed,opacity=.4] (S-1-i-1) to[right vertical torus arc] (S-1-i-2);
\end{scope}
\begin{scope}[blue,very thick]
\draw (S-1-r) to[arc with diameter={170:10:1 and .8}] (S-2-l);
\draw[dashed,opacity=.4] (S-1-r) to[arc with diameter={170:370:1 and .8}] (S-2-l);
\end{scope}
\node[red] at ($(S-1-i-2)+(0,-.4)$) {$\beta$};
\node[blue] at ($(S-2-l)+(.2,.3)$) {$\alpha$};
\end{tikzpicture}
\end{center}

Mediante il criterio del bigono è facile vedere che $\alpha$ e $\beta$ sono in posizione minimale, dunque $i(a,b)=2$.
\item\textbf{Supponiamo che $\beta$ non sia separante}. In questo caso esiste una curva $\alpha$ che interseca $\beta$ esattamente una volta.

\begin{center}
\begin{tikzpicture}
\pic[surface color,thick] {closed surface={radius=1.5cm,name=S,g=2}};
\begin{scope}[red,very thick]
\draw (S-1-b-1) to[left vertical torus arc] (S-1-b-2);
\draw[dashed,opacity=.4] (S-1-b-1) to[right vertical torus arc] (S-1-b-2);
\end{scope}
\draw[blue,very thick] ($(S-1-t-1)!.3!(S-1-t-2)$) to[arc with diameter={90:270:{1/cos(70)} and 1}] ($(S-1-b-1)!.3!(S-1-b-2)$) to[out=0,in=180] ($(S-1-i-2)!.4!(S-1-i-1)$) to[out=0,in=180] ($(S-2-b-1)!.3!(S-2-b-2)$) to[arc with diameter={-90:90:{1/cos(70)} and 1}] ($(S-2-t-1)!.3!(S-2-t-2)$) to[out=180,in=0] ($(S-1-i-1)!.15!(S-1-i-2)$) to[out=180,in=0] ($(S-1-t-1)!.3!(S-1-t-2)$);
\node[red] at ($(S-1-b-2)+(0,-.4)$) {$\beta$};
\node[blue] at ($(S-l)+(-.2,.5)$) {$\alpha$};
\end{tikzpicture}
\end{center}

Prendendo come $a$ la classe di isotopia di $\alpha$, otteniamo che $i(a,b)=1$.
\end{itemize}
Dal \thref{intersection-number-dehn-twist} sappiamo che
\[
i(a,T_b^k(a))=k\cdot i(a,b)^2
\]
per ogni $k>0$, dunque in particolare $T_b^k(a)\neq a$. Pertanto l'azione di $T_b^k$ su $\mathcal{S}$ è non banale, e in particolare $T_b$ non è banale.

\subsection*{Esercizio 3.2}
\begin{lemma}\thlabel{existence-discerning-curve}
Siano $a,b\in\mathcal{S}$ due classi di isotopia non banali distinte. Allora esiste una classe $c\in\mathcal{S}$ tale che $i(a,c)\neq i(b,c)$.
\end{lemma}
\begin{proof}

\end{proof}
\begin{enumerate}[(1)]
\item \begin{itemize}
\item Cominciamo a mostrare che la classe di isotopia di $T_\alpha$ non dipende dall'orientazione di $\alpha$. Sia dunque $\map{\alpha}{S^1}{S}$ una curva semplice chiusa, e sia $\map{\overline\alpha}{S^1}{S}$ la curva inversa, ossia quella definita da $\overline\alpha(e^{ix})=\alpha(e^{-ix})$. Se $\map{\psi}{A}{S}$ è un intorno regolare orientato di $\alpha$, allora
\Map{\overline\psi}{A}{S}{(e^{ix},t)}{\psi(e^{-ix},-t)}
è un intorno regolare orientato di $\overline\alpha$. Poiché la classe di isotopia dei Dehn twist non dipende dalla scelta di $f$, non è restrittivo supporre che $f(-t)=2\pi-f(t)$. Mostriamo allora che $T_\alpha(p)=T_{\overline\alpha}(p)$ per ogni $p\in S$ (dunque in particolare sono isotopi). La tesi è ovvia per $p\not\in\psi(A)$, dunque supponiamo $p\in\psi(A)$; è sufficiente far vedere che $\psi\circ\theta\circ\psi^{-1}\circ\overline\psi=\overline\psi\circ\theta$. Effettivamente:
\begin{align*}
(\psi\circ\theta\circ\psi^{-1}\circ\overline\psi)(e^{ix},t)&=\psi(\theta(e^{-ix},-t))=\psi(e^{i(-x+f(-t))},-t)=\psi(e^{-i(x+f(t))},-t);\\
(\overline\psi\circ\theta)(e^{ix},t)&=\overline\psi(e^{i(x+f(t))},t)=\psi(e^{-i(x+f(t))},-t).
\end{align*}
\item Supponiamo che le curve semplici chiuse non banali $\map{\alpha,\beta}{S^1}{S}$ siano due rappresentanti della stessa classe di isotopia, ossia che esista un'isotopia (ambiente) $\map{F}{S\times[0,1]}{S}$ tale che $F_0=\id_S$ e $F_1\circ\alpha=\beta$. Osserviamo che, per quanto abbiamo dimostrato, possiamo orientare $\alpha$ e $\beta$ in modo che una tale isotopia esista. Sia $\map{\psi}{A}{S}$ un intorno regolare orientato di $\alpha$; notiamo che $F_1\circ\psi$ è un intorno regolare orientato di $\beta$. È allora evidente che un Dehn twist intorno a $\beta$ è dato da $T_\beta=F_1\circ T_\alpha\circ F_1^{-1}$, che è ovviamente isotopo a $T_\alpha$. Questo mostra che $T_\beta=T_\alpha$ in $\MCG(S)$.
\item Siano ora $\map{\alpha,\beta}{S^1}{S}$ due curve semplici chiuse non banali e non isotope, e siano $a,b\in\mathcal{S}$ le corrispondenti classi di isotopia. Per il \thref{existence-discerning-curve}, esiste una classe $c\in\mathcal{S}$ tale che $i(a,c)\neq i(b,c)$. Dal \thref{intersection-number-dehn-twist} otteniamo
\[
i(c,T_a(c))=i(c,a)^2\neq i(c,b)^2=i(c,T_b(c)),
\]
da cui $T_a\neq T_b$ come elementi di $\MCG(S)$.
\end{itemize}
\item Per non creare conflitti di notazione, siano $h\in\MCG(S)$, $a\in\mathcal{S}$. Sia $\map{\alpha}{S^1}{S}$ un rappresentante di $a$, e con lieve abuso di notazione trattiamo $h$ come un diffeomorfismo di $S$. Sia $\map{\psi}{A}{S}$ un intorno regolare di $\alpha$: notiamo che $h\circ\psi$ è un intorno regolare di $h\circ\alpha$. Ma allora è evidente che $h\circ T_\alpha\circ h^{-1}$ è un Dehn twist intorno a $h\circ\alpha$, da cui la tesi.
\end{enumerate}

\subsection*{Esercizio 3.5}
\newcommand{\ngon}[3][0]{
\tikzmath{
\g = #2;
\r = #3;
\n = \g * 4;
\twog = \g * 2;
}
\foreach \i [evaluate=\i as \an using 180*((2*\i-1)/\n+1)] in {1,...,\n} {
    \tkzDefPoint(\an:\r){x_\i}
    \tikzmath{\an1 = \an + #1;}
    \tkzLabelPoint[label=\an1:$x_{\i}$,anchor=base](x_\i){}
}
\begin{scope}[on background layer]
\tkzFillPolygon[blue!10](x_1,x_...,x_\n)
\end{scope}
\foreach \i in {1,...,\twog} {
    \tikzmath{
        \j = \i + 1;
        \l = \twog + \i;
        \k = \l + 1;
        if \k > \n then {
            \k = 1;
        };
        if Mod(\i, 2) == 0 then {
            let \col = white;
        } else {
            let \col = black;
        };
        \s = .15 * 3 / (\r * 5.1 * sin(180 / \n));
        \t = (\i + Mod(\i, 2) - 2) / 2;
    }
    \foreach \b/\e in {\i/\j,\k/\l} {
        \tkzDrawSegment[postaction={decorate},decoration={markings,mark=between positions (0.5+\s*0.75-\s/2*\t) and (0.51+\s*0.75+\s/2*\t) step \s with {\arrow[scale=1.2,\col]{Triangle}\arrow[scale=1.2,black]{Triangle[open]}}}](x_\b,x_\e)
    }
}
\tkzDrawPoints(x_1,x_...,x_\n)
}

Consideriamo la superficie chiusa ottenuta incollando lati opposti di un $4g$-gono regolare con orientazioni parallele. Più precisamente, detti $x_1,\ldots,x_{4g}$ i vertici del poligono regolare, incolliamo il segmento $x_ix_{i+1}$ con il segmento $x_{2g+i+1}x_{2g+i}$. La superficie $\Sigma$ così ottenuta ha una struttura di CW-complesso con una $0$-cella, $2g$ $1$-celle e una $2$-cella.
\begin{center}
\begin{tikzpicture}
\ngon{3}{3}
\begin{scope}[on background layer]
\foreach \i [evaluate=\i as \j using \i-1] in {4,...,\n} {
    \tkzDrawSegment[thin,gray](x_1,x_\j)
}
\end{scope}
\end{tikzpicture}
\end{center}
\begin{itemize}
\item\textbf{$\Sigma$ è orientabile.} Questo si vede immediatamente triangolando il poligono regolare e osservando che le identificazioni fra lati invertono l'orientazione.
\item\textbf{Una base per $H_1(\Sigma,\ZZ)$ è data dai segmenti $x_ix_{i+1}$ per $1\le i<2g$.} Questo si vede facilmente calcolando l'omologia cellulare: infatti i segmenti $x_ix_{i+1}$ sono esattamente le $1$-celle, e hanno tutte bordo nullo. Al contempo, anche l'unica $2$-cella ha bordo nullo, dunque $H_1(\Sigma,\ZZ)\iso\ZZ^{2g}$ con base data dalle $1$-celle.
\end{itemize}
In particolare, $\Sigma$ è una superficie chiusa orientabile di genere $g$.

Per ogni $1\le i<2g$, sia $\alpha_i\in H_1(\Sigma,\ZZ)$ la classe rappresentata in omologia dal segmento $x_ix_{i+1}$. Consideriamo l'automorfismo $\map{f}{\Sigma}{\Sigma}$ indotto dalla rotazione di angolo $\pi$ intorno al centro del poligono regolare. Osserviamo che il segmento $x_ix_{i+1}$ viene mandato da $f$ nel segmento $x_{2g+i}x_{2g+i+1}$, dunque $f_*(\alpha_i)=-\alpha_i$. Poiché gli $\alpha_i$ formano una base di $H_1(\Sigma,\ZZ)$, abbiamo che $f_*=-\id_{H_1(\Sigma,\ZZ)}$. Ovviamente $f$ ha ordine $2$ e preserva l'orientazione, dunque $[f]\in\MCG(\Sigma)$ è l'involuzione iperellittica cercata.

\begin{center}
\begin{tikzpicture}
\begin{scope}[xshift=-4cm]
\ngon{2}{2}
\tkzDefPoint(0,0){o}
\tkzDrawPoint(o)
\tkzDefMidPoint(x_1,x_2)\tkzGetPoint{u_0}
\tkzDefPointWith[linear, K=.3](o,u_0)\tkzGetPoint{u_1}
\tkzDrawArc[thick,rotate,->,>=Triangle,orange!50!red](o,u_1)(180)
\begin{scope}[on background layer]
\tkzFillPolygon[orange!50](o,x_1,x_2)
\end{scope}
\end{scope}

\begin{scope}[xshift=4cm,rotate=180]
\ngon[180]{2}{2}
\tkzDefPoint(0,0){o}
\tkzDrawPoint(o)
\begin{scope}[on background layer]
\tkzFillPolygon[orange!50](o,x_1,x_2)
\end{scope}
\end{scope}
\draw[thick,->,orange!50!red] (-1,0) -- node[above]{$f$} (1,0);
\end{tikzpicture}
\end{center}

\subsection*{Esercizio 3.6}
Siano $[f]\in\MCG(S_g)$, $[m]\in\Teich(S_g)$ tali che $[f_*m]=[m]$. Ciò significa che esiste un diffeomorfismo $h$ di $S_g$ isotopo all'identità e tale che $f_*m=h_*m$. Ma allora $(h^{-1}\circ f)_*m=m$; poiché $h^{-1}\circ f$ e $f$ sono isotopi, essi rappresentano la stessa classe in $\MCG(S_g)$, dunque possiamo supporre (a meno di cambiare rappresentante) che $f_*m=m$. Ciò significa precisamente che $f$ è un'isometria per la superficie $S_g$ con la metrica $m$.

I punti singolari dello spazio dei moduli sono precisamente le (classi di) metriche che sono fissate da elementi non banali di $\MCG(S_g)$. Come abbiamo visto, se elemento $\varphi\in\MCG(S_g)$ fissa una classe $[m]\in\Teich(S_g)$, allora esiste un rappresentante $f$ di $\varphi$ (che non sarà isotopo all'identità se $\varphi$ è non banale) che è un'isometria per $S_g$ munita della metrica $m$.