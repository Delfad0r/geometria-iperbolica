\section*{Esercizi dell'11 aprile}

\subsection*{Esercizio 3.2}
Per fissare la notazione, ricordiamo come si definiscono i Dehn twist. Sia $\map{\alpha}{S^1}{S}$ una curva semplice chiusa. Sia $A=S^1\times[-1,1]$ con l'orientazione indotta da quelle standard di $S^1$ e $[0,1]$. Fissiamo un intorno regolare di $\alpha$, ossia un embedding $\map{\psi}{A}{S}$ tale che $\psi(x,0)=\alpha(x)$ per ogni $x\in S^1$; scegliamolo in modo che $\psi$ preservi l'orientazione. Fissiamo infine una funzione $\map{f}{[-1,1]}{[0,2\pi]}$ tale che
$f(t)=0$ per $t\le-\frac{1}{2}$ e $f(t)=2\pi$ per $t\ge\frac{1}{2}$, e definiamo
\Map{\theta}{A}{A}{(e^{ix},t)}{(e^{i(x+f(t))},t).}
Possiamo ora definire il Dehn twist intorno a $\alpha$ come il diffeomorfismo $\map{T_\alpha}{S}{S}$ tale che
\[
T_\alpha(p)=\begin{cases}
p&p\not\in\psi(A)\\
(\psi\circ\theta\circ\psi^{-1})(p)&p\in\psi(A).
\end{cases}
\]
Come visto a lezione, la classe di isotopia di $T_\alpha$ non dipende dalla scelta dell'intorno regolare $\psi$ né della funzione $f$.
\begin{enumerate}[(1)]
\item \begin{itemize}
\item Cominciamo a mostrare che la classe di isotopia di $T_\alpha$ non dipende dall'orientazione di $\alpha$. Sia dunque $\map{\alpha}{S^1}{S}$ una curva semplice chiusa, e sia $\map{\overline\alpha}{S^1}{S}$ la curva inversa, ossia quella definita da $\overline\alpha(e^{ix})=\alpha(e^{-ix})$. Se $\map{\psi}{A}{S}$ è un intorno regolare orientato di $\alpha$, allora
\Map{\overline\psi}{A}{S}{(e^{ix},t)}{\psi(e^{-ix},-t)}
è un intorno regolare orientato per $\overline\alpha$. Poiché la classe di isotopia dei Dehn twist non dipende dalla scelta di $f$, non è restrittivo supporre che $f(-t)=2\pi-f(t)$. Mostriamo allora che $T_\alpha(p)=T_{\overline\alpha}(p)$ per ogni $p\in S$ (dunque in particolare sono isotopi). La tesi è ovvia per $p\not\in\psi(A)$, dunque supponiamo $p\in\psi(A)$; è sufficiente far vedere che $\psi\circ\theta\circ\psi^{-1}\circ\overline\psi=\overline\psi\circ\theta$. Effettivamente:
\begin{align*}
(\psi\circ\theta\circ\psi^{-1}\circ\overline\psi)(e^{ix},t)&=\psi(\theta(e^{-ix},-t))=\psi(e^{i(-x+f(-t))},-t)=\psi(e^{-i(x+f(t))},-t);\\
(\overline\psi\circ\theta)(e^{ix},t)&=\overline\psi(e^{i(x+f(t))},t)=\psi(e^{-i(x+f(t))},-t).
\end{align*}
\item Supponiamo che le curve semplici chiuse $\map{\alpha,\beta}{S^1}{S}$ siano due rappresentanti della stessa classe di isotopia, ossia che esista un'isotopia (ambiente) $\map{F}{S\times[0,1]}{S}$ tale che $F_0=\id_S$ e $F_1\circ\alpha=\beta$. Osserviamo che, per quanto abbiamo dimostrato, possiamo orientare $\alpha$ e $\beta$ in modo che una tale isotopia esista. Sia $\map{\psi}{A}{S}$ un intorno regolare orientato di $\alpha$; notiamo che $F_1\circ\psi$ è un intorno regolare orientato di $\beta$. È allora evidente che un Dehn twist intorno a $\beta$ è dato da $T_\beta=F_1\circ T_\alpha\circ F_1^{-1}$, che è ovviamente isotopo a $T_\alpha$. Questo mostra che $T_\beta=T_\alpha$ in $\MCG(S)$.
\item 
\end{itemize}
\end{enumerate}

\subsection*{Esercizio 3.5}
\usetikzlibrary{backgrounds}
\usetikzlibrary{math}
\newcommand{\ngon}[3][0]{
\tikzmath{
\g = #2;
\r = #3;
\n = \g * 4;
\twog = \g * 2;
}
\foreach \i [evaluate=\i as \an using 180*((2*\i-1)/\n+1)] in {1,...,\n} {
    \tkzDefPoint(\an:\r){x_\i}
    \tikzmath{\an1 = \an + #1;}
    \tkzLabelPoint[label=\an1:$x_{\i}$,anchor=base](x_\i){}
}
\begin{scope}[on background layer]
\tkzFillPolygon[blue!10](x_1,x_...,x_\n)
\end{scope}
\foreach \i in {1,...,\twog} {
    \tikzmath{
        \j = \i + 1;
        \l = \twog + \i;
        \k = \l + 1;
        if \k > \n then {
            \k = 1;
        };
        if Mod(\i, 2) == 0 then {
            let \col = white;
        } else {
            let \col = black;
        };
        \s = .15 * 3 / (\r * 5.1 * sin(180 / \n));
        \t = (\i + Mod(\i, 2) - 2) / 2;
    }
    \foreach \b/\e in {\i/\j,\k/\l} {
        \tkzDrawSegment[postaction={decorate},decoration={markings,mark=between positions (0.5+\s*0.75-\s/2*\t) and (0.51+\s*0.75+\s/2*\t) step \s with {\arrow[scale=1.2,\col]{Triangle}\arrow[scale=1.2,black]{Triangle[open]}}}](x_\b,x_\e)
    }
}
\tkzDrawPoints(x_1,x_...,x_\n)
}

Consideriamo la superficie chiusa ottenuta incollando lati opposti di un $4g$-gono regolare con orientazioni parallele. Più precisamente, detti $x_1,\ldots,x_{4g}$ i vertici del poligono regolare, incolliamo il segmento $x_ix_{i+1}$ con il segmento $x_{2g+i+1}x_{2g+i}$. La superficie $\Sigma$ così ottenuta ha una struttura di CW-complesso con una $0$-cella, $2g$ $1$-celle e una $2$-cella.
\begin{figure}[h!]
\centering
\begin{tikzpicture}
\ngon{3}{3}
\begin{scope}[on background layer]
\foreach \i [evaluate=\i as \j using \i-1] in {4,...,\n} {
    \tkzDrawSegment[thin,gray](x_1,x_\j)
}
\end{scope}
\end{tikzpicture}
\end{figure}
\begin{itemize}
\item\textbf{$\Sigma$ è orientabile.} Questo si vede immediatamente triangolando il poligono regolare e osservando che le identificazioni fra lati invertono l'orientazione.
\item\textbf{Una base per $H_1(\Sigma,\ZZ)$ è data dai segmenti $x_ix_{i+1}$ per $1\le i<2g$.} Questo si vede facilmente calcolando l'omologia cellulare: infatti i segmenti $x_ix_{i+1}$ sono esattamente le $1$-celle, e hanno tutte bordo nullo. Al contempo, anche l'unica $2$-cella ha bordo nullo, dunque $H_1(\Sigma,\ZZ)\iso\ZZ^{2g}$ con base data dalle $1$-celle.
\end{itemize}
In particolare, $\Sigma$ è una superficie chiusa orientabile di genere $g$.

Per ogni $1\le i<2g$, sia $\alpha_i\in H_1(\Sigma,\ZZ)$ la classe rappresentata in omologia dal segmento $x_ix_{i+1}$. Consideriamo l'automorfismo $\map{f}{\Sigma}{\Sigma}$ indotto dalla rotazione di angolo $\pi$ intorno al centro del poligono regolare. Osserviamo che il segmento $x_ix_{i+1}$ viene mandato da $f$ nel segmento $x_{2g+i}x_{2g+i+1}$, dunque $f_*(\alpha_i)=-\alpha_i$. Poiché gli $\alpha_i$ formano una base di $H_1(\Sigma,\ZZ)$, abbiamo che $f_*=-\id_{H_1(\Sigma,\ZZ)}$. Ovviamente $f$ ha ordine $2$ e preserva l'orientazione, dunque $[f]\in\MCG(\Sigma)$ è l'involuzione iperellittica cercata.

\begin{figure}[h!]
\centering
\begin{tikzpicture}
\begin{scope}[xshift=-4cm]
\ngon{2}{2}
\tkzDefPoint(0,0){o}
\tkzDrawPoint(o)
\tkzDefMidPoint(x_1,x_2)\tkzGetPoint{u_0}
\tkzDefPointWith[linear, K=.3](o,u_0)\tkzGetPoint{u_1}
\tkzDrawArc[thick,rotate,->,>=Triangle,orange!50!red](o,u_1)(180)
\begin{scope}[on background layer]
\tkzFillPolygon[orange!50](o,x_1,x_2)
\end{scope}
\end{scope}

\begin{scope}[xshift=4cm,rotate=180]
\ngon[180]{2}{2}
\tkzDefPoint(0,0){o}
\tkzDrawPoint(o)
\begin{scope}[on background layer]
\tkzFillPolygon[orange!50](o,x_1,x_2)
\end{scope}
\end{scope}
\draw[thick,->,orange!50!red] (-1,0) -- node[above]{$f$} (1,0);
\end{tikzpicture}
\end{figure}

\newpage
\subsection*{Esercizio 3.6}
Siano $[f]\in\MCG(S_g)$, $[m]\in\Teich(S_g)$ tali che $[f_*m]=[m]$. Ciò significa che esiste un diffeomorfismo $h$ di $S_g$ isotopo all'identità e tale che $f_*m=h_*m$. Ma allora $(h^{-1}\circ f)_*m=m$; poiché $h^{-1}\circ f$ e $f$ sono isotopi, essi rappresentano la stessa classe in $\MCG(S_g)$, dunque possiamo supporre (a meno di cambiare rappresentante) che $f_*m=m$. Ciò significa precisamente che $f$ è un'isometria per la superficie $S_g$ con la metrica $m$.

I punti singolari dello spazio dei moduli sono precisamente le (classi di) metriche che sono fissate da elementi non banali di $\MCG(S_g)$. Come abbiamo visto, se elemento $\varphi\in\MCG(S_g)$ fissa una classe $[m]\in\Teich(S_g)$, allora esiste un rappresentante $f$ di $\varphi$ (che non sarà isotopo all'identità se $\varphi$ è non banale) che è un'isometria per $S_g$ munita della metrica $m$.