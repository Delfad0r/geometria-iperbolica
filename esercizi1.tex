\section*{Esercizi del 14 marzo}

\subsection*{Esercizio 1.4}
Sia $K>0$ tale che le palle iperboliche di raggio $K$ abbiano area maggiore di $2\pi$; mostriamo che tale $K$ soddisfa la condizione richiesta.

Sia $\Delta\subs\HH^2$ un triangolo, $p\in\Delta$ un punto giacente su un lato $\ell$. Se $p$ è un vertice la tesi è ovvia, dunque supponiamo che non lo sia. Sia $\mathcal{A}$ il semipiano (aperto) delimitato da $\ell$ su cui giace il triangolo. Definiamo $B\subs\HH^2$ come l'intersezione di $\mathcal{A}$ con la palla di centro $p$ e raggio $K$; osserviamo che $B$ ha area maggiore di $2\pi/2=\pi$. Di conseguenza, essendo l'area di $\Delta$ al più $\pi$, necessariamente $B$ non è contenuto in $\Delta$. Poiché $B$ interseca $\Delta$ ed è connesso, deve esistere un punto di $q\in B$ che giace sul bordo di $\Delta$. Giacché $\mathcal{A}$ e $\ell$ sono disgiunti, $q$ deve appartenere a uno degli altri due lati; essendo la distanza fra $p$ e $q$ al più $K$, otteniamo la tesi.

\subsection*{Esercizio 1.5}
Utilizziamo il modello $H^2=\{(x,y)\in\RR^2:y>0\}$ del semipiano.
Siano $\ell_1$, $\ell_2$, $\ell_3$ i lati del triangolo $\Delta$, con $\ell_1$ di lunghezza $l$. A meno di isometria, possiamo supporre che $\ell_1$ giaccia sulla circonferenza centrata in $(0,0)$ di raggio $r>0$, e che $\ell_2$ giaccia su una retta verticale (parallela all'asse $y$). Siano inoltre $\alpha$, $\beta$ gli angoli che le rette congiungenti $(0,0)$ agli estremi di $\ell_1$ formano con l'asse $x$.

Una parametrizzazione di $\ell_1$ è data dalla curva
\Map{\gamma}{[\alpha,\pi-\beta]}{H^2}{t}{r(\cos t,\sin t).}

Pertanto la lunghezza di $\ell_1$ si può calcolare come
\[
l=\int_\alpha^{\pi-\beta}\frac{1}{r\sin t}\norm{\gamma'}_Edt=\int_\alpha^{\pi-\beta}\frac{1}{r\sin t}\cdot rdt=\int_\alpha^{\pi-\beta}\frac{1}{\sin t}dt>\pi-\beta-\alpha,
\]
dove $\norm{-}_E$ indica la norma euclidea. Denotiamo con $\mathcal{A}$ il cono sopra $\ell_1$ di vertice $\infty$, ossia
\[
\mathcal{A}=\{(x,y)\in H^2:-r\cos\beta\le x\le r\cos\alpha,x^2+y^2\ge r^2\}.
\]
È evidente che $\Delta$ è contenuto in $\mathcal{A}$; inoltre $\mathcal{A}$ è un triangolo con angoli $\alpha$, $\beta$ e $0$, dunque ha area $\pi-\alpha-\beta$. Ma allora
\[
l>\pi-\alpha-\beta=\Area(\mathcal{A})\ge\Area(\Delta).
\]


\begin{tikzpicture}[scale=2]
\tkzDefPoint(0,0){O}
\tkzDefPoint(1,0){K}
\tkzDefPoint(0.6,0){U0}
\tkzDefPointWith[orthogonal](U0,O)\tkzGetPoint{U1}
\tkzInterLC(U0,U1)(O,K)\tkzGetFirstPoint{U}
\tkzDefPoint(-0.8,0){V0}
\tkzDefPointWith[orthogonal](V0,O)\tkzGetPoint{V1}
\tkzInterLC(V0,V1)(O,K)\tkzGetSecondPoint{V}
\tkzDefPointOnLine[pos=4](V0,V)\tkzGetPoint{W}
\tkzDefLine[mediator](U,W)\tkzGetPoints{W0}{W1}
\tkzInterLL(W0,W1)(O,K)\tkzGetPoint{O1}
\tkzDefPointWith[orthogonal,K=-.4](U,O)\tkzGetPoint{U2}
\tkzDefPointWith[orthogonal,K=.4](V,O)\tkzGetPoint{V2}
\tkzDefShiftPoint[W](0,0.5){W3}
\tkzDefPointWith[colinear=at W3](V0,U0)\tkzGetPoint{U3}
\tkzFillPolygon[green!10](V0,U0,U3,W3)
\tkzFillSector[white,rotate](O,K)(180)
\tkzFillAngles[size=.2,fill=blue!25](U0,O,U U3,U,U2)
\tkzMarkAngles[size=.2](U0,O,U U3,U,U2)
\tkzFillAngles[size=.2,fill=red!25](V,O,V0 V2,V,V1)
\tkzMarkAngles[size=.2,mark=||](V,O,V0 V2,V,V1)
\tkzDrawSemiCircle[gray](O,K)
\tkzDrawSegment[gray](V0,W3)
\tkzDrawSegment[gray](U0,U3)
\tkzDrawLine[gray,add=0.5 and 0.5](V0,U0)
\tkzDrawArc[black,line width=.5pt](O1,U)(W)
\tkzDrawSegment[black,line width=.5pt](V,W)
\tkzDrawArc[black,line width=.5pt,postaction={decorate},decoration={markings,mark=between positions 0.2 and 0.8 step 0.2 with {\arrow[thick]{stealth}}}](O,U)(V)
\tkzDrawSegments[gray](O,U O,V)
\tkzDrawSegments[gray](U,U2 V,V2)
\tkzDrawPoint[black!70](O)
\tkzDrawPoints[black](U,V,W)
\tkzLabelSegment(V,W){$\ell_2$}
\tkzLabelCircle[above](O,K)(100){$\ell_1$}
\tkzLabelCircle[above](O1,U)(30){$\ell_3$}
\tkzLabelSegment[black!70,above](O,U){$r$}
\tkzLabelAngles[blue,pos=.35](U0,O,U){$\alpha$}
\tkzLabelAngles[red,pos=.35](V,O,V0){$\beta$}
\tkzLabelSegment[pos=.9,green!40!black](U,U3){$\mathcal{A}$}
\end{tikzpicture}



\subsection*{Esercizio 1.9}
\begin{lemma*}
Siano $A\in O(n)$ una matrice ortogonale, $b\in\RR^n$ un vettore. Definiamo $\varphi\in\Isom(\RR^n)$ come $\varphi(x)=Ax+b$. Supponiamo che $\varphi$ non abbia punti fissi. Allora esiste una retta affine di $\RR^n$ che è $\varphi$-invariante.
\end{lemma*}
\begin{proof}\leavevmode
\begin{itemize}
\item\textbf{Vale $\RR^n=\ker(A-I)\dirsum\im(A-I)$}.\\
Per motivi di dimensione, è sufficiente mostrare che i due sottospazi hanno intersezione banale. Sia $v\in\ker(A-I)\cup\im(A-I)$; allora $v=Aw-w$ per un qualche $w\in\RR^n$. Allora
\[
\scal{v}{v}=\scal{Aw-w}{v}=\scal{Aw}{v}-\scal{w}{v}=\scal{w,Av}-\scal{w}{v}=\scal{w}{Av-v}=0,
\]
da cui $v=0$.
\item\textbf{Vale $b\not\in\im(A-I)$.}\\
Se per assurdo $b=Aw-w$, allora $\varphi(-w)=-Aw+b=-w$, dunque $\varphi$ avrebbe un punto fisso, il che è contro l'ipotesi.
\item\textbf{L'isometria $\varphi$ ammette una retta invariante.}\\
Utilizzando la decomposizione del primo punto, scriviamo $b=v+(Aw-w)$ con $Av=v$. Poiché $b\not\in\im(A-I)$, necessariamente $v\neq 0$. Mostriamo che la retta affine $\ell=-w+\spa(v)$ è $\varphi$-invariante. Per $t\in\RR$ vale
\[
\varphi(-w+tv)=-Aw+tAv+b=-Aw+tv+(v+Aw-w)=-w+(t+1)v\in\ell,
\]
da cui la tesi.\qedhere
\end{itemize}
\end{proof}

Sia $\psi\in\Isom(\HH^n)$ un'isometria parabolica. Consideriamo il modello del semispazio $H^n$; possiamo supporre che $\psi$ fissi $\infty$. Allora $\psi$ si scrive come $\psi(x,t)=(Ax+b,t)$ per opportuni $A\in O(n-1)$, $b\in\RR^{n-1}$. Per il Lemma, esiste una retta euclidea $\ell=w+\spa(v)\subs\RR^{n-1}$ invariante per la mappa $(x\mapsto Ax+b)$. È allora evidente che il piano iperbolico $\{(w+sv,t):s\in\RR,t>0\}\subs H^n$ è $\psi$-invariante.